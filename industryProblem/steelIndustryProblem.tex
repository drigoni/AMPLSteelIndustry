\documentclass[12pt]{article} % Default font size is 12pt, it can be changed here

\usepackage{geometry} % Required to change the page size to A4
\geometry{a4paper} % Set the page size to be A4 as opposed to the default US Letter

\usepackage{graphicx} % Required for including pictures

\usepackage{float} % Allows putting an [H] in \begin{figure} to specify the exact location of the figure

\usepackage{wrapfig} % Allows in-line images such as the example fish picture

\usepackage[italian]{babel}
\usepackage[utf8]{inputenc}
\usepackage[T1]{fontenc}
\usepackage{lmodern}
\usepackage{listings}
\usepackage{lstautogobble}


\linespread{1.2} % Line spacing

\lstset{
  basicstyle=\ttfamily\scriptsize,
  breaklines=true,
  autogobble=true,
  frame=single,
inputencoding=utf8,
            extendedchars=true,
            literate=%
            {é}{{\'{e}}}1
            {è}{{\`{e}}}1
            {ê}{{\^{e}}}1
            {ë}{{\¨{e}}}1
            {û}{{\^{u}}}1
            {ù}{{\`{u}}}1
            {â}{{\^{a}}}1
            {à}{{\`{a}}}1
            {î}{{\^{i}}}1
            {ì}{{\'{i}}}1
            {ô}{{\^{o}}}1
            {ç}{{\c{c}}}1
            {Ç}{{\c{C}}}1
            {É}{{\'{E}}}1
            {Ê}{{\^{E}}}1
            {À}{{\`{A}}}1
            {Â}{{\^{A}}}1
            {Î}{{\^{I}}}1
}



%----------------------------------------------------------------------------------------
%	TITLE PAGE
%----------------------------------------------------------------------------------------

\title{\Huge \textbf{AMPLSteelIndustryProblem} \\ \Huge Un problema di Programmazione Lineare }
% Author
\author{\textsc{Davide Rigoni}}

%----------------------------------------------------------------------------------------
%	TABLE OF CONTENTS
%----------------------------------------------------------------------------------------
\begin{document}

\maketitle
\newpage

\tableofcontents % Include a table of contents
\listoftables

\newpage

%----------------------------------------------------------------------------------------
% FORMULAZIONE PROBLEMA
%----------------------------------------------------------------------------------------
\section{Formulazione del problema}
Una industria metalmeccanica possiede due fabbriche che grazie all'utilizzo di macchine apposite, producono tre tipi di acciaio diverso per soddisfare le richieste del territorio.

La prima fabbrica F\ped{1} produce acciaio normale, inox e temprato con l'utilizzo di quattro  macchine: M\ped{1}, M\ped{2}, M\ped{3} e M\ped{4}.

La seconda fabbrica F\ped{2} produce solamente acciaio normale e temprato con l'utilizzo di due macchine: M\ped{1}, M\ped{2}.

Per evitare i lunghi tempi di avvio delle macchine, l'industria le fa lavorare 24 ore su 24 dividendo la giornata in tre turni lavorativi: 8-16, 16-24 e 24-8. 

Le tabelle seguenti riportano per ogni turno di lavoro e per ogni macchina il tempo necessario in ore per lavorare una tonnellata di acciaio nella prima fabbrica.

\begin{table}[!htbp]
  \centering
  \begin{tabular}{ | c | c | c | c | c | }
    \hline
    \textbf{Acciaio} & \textbf{M\ped{1}} & \textbf{M\ped{2}} & \textbf{M\ped{3}} & \textbf{M\ped{4}} \\
    \hline
    Normale & 0.9 & 1.1 & 0.9 & 0.8 \\
    Inox & 1.5 & 1.6 & 1.4 & 1.4 \\
    Temprato & 1.1 & 1 & 1.1 & 1.1 \\
    \hline
  \end{tabular}
  \caption{Turno di lavoro 8-16 fabbrica 1}
\end{table}
\begin{table}[!htbp]
  \centering
  \begin{tabular}{ | c | c | c | c | c | }
    \hline
    \textbf{Acciaio} & \textbf{M\ped{1}} & \textbf{M\ped{2}} & \textbf{M\ped{3}} & \textbf{M\ped{4}} \\
    \hline
    Normale & 1 & 1 & 0.9 & 0.9 \\
    Inox & 1.5 & 1.5 & 1.5 & 1.3 \\
    Temprato & 1 & 1 & 1 & 1 \\
    \hline
  \end{tabular}
  \caption{Turno di lavoro 16-24 fabbrica 1} 
\end{table}
\begin{table}[!htbp]
  \centering
  \begin{tabular}{ | c | c | c | c | c | }
    \hline
    \textbf{Acciaio} & \textbf{M\ped{1}} & \textbf{M\ped{2}} & \textbf{M\ped{3}} & \textbf{M\ped{4}} \\
    \hline
    Normale & 1.3 & 1.2 & 1 & 1 \\
    Inox & 1.7 & 1.7 & 1.3 & 1.3 \\
    Temprato & 1.4 & 1.3 & 1.2 & 1.1\\
    \hline
  \end{tabular}
  \caption{Turno di lavoro 24-8 fabbrica 1} 
\end{table}

\newpage

Le tabelle seguenti riportano per ogni turno di lavoro e per ogni macchina il tempo necessario in ore per lavorare una tonnellata di acciaio nella seconda fabbrica.

\begin{table}[!htbp]
  \centering
  \begin{tabular}{ | c | c | c | c | }
    \hline
    \textbf{Acciaio} & \textbf{M\ped{1}} & \textbf{M\ped{2}} & \textbf{M\ped{3}}  \\
    \hline
    Normale & 1 & 1 & 1 \\
    Temprato & 1.2 & 1.3 & 1.1 \\
    \hline
  \end{tabular}
  \caption{Turno di lavoro 8-16 fabbrica 2}
\end{table}
\begin{table}[!htbp]
  \centering
  \begin{tabular}{ | c | c | c | c | }
    \hline
    \textbf{Acciaio} & \textbf{M\ped{1}} & \textbf{M\ped{2}} & \textbf{M\ped{3}}  \\
    \hline
    Normale & 1.1 & 0.8 & 0.9 \\
    Temprato & 1 & 1.3 & 1.3 \\
    \hline
  \end{tabular}
  \caption{Turno di lavoro 16-24 fabbrica 2} 
\end{table}
\begin{table}[!htbp]
  \centering
  \begin{tabular}{ | c | c | c | c | }
    \hline
    \textbf{Acciaio} & \textbf{M\ped{1}} & \textbf{M\ped{2}} & \textbf{M\ped{3}} \\
    \hline
    Normale & 1 & 1.2 & 1 \\
    Temprato & 1 & 1.1 & 1 \\
    \hline
  \end{tabular}
  \caption{Turno di lavoro 24-8 fabbrica 2} 
\end{table}

Tuttavia, per questioni sindacali, in entrambe le fabbriche ciascuna macchina può lavorare per un tempo limitato e dipendente dai turni di lavoro degli operai, come riportato nelle tabelle che seguono:

\begin{table}[!htbp]
  \centering
  \begin{tabular}{ | c | c | c | c | }
    \hline
    \textbf{Macchina} & \textbf{8-16} & \textbf{16-24} & \textbf{24-8} \\
    \hline
    M\ped{1} & 7 & 6 & 6 \\
    M\ped{2} & 7 & 7 & 6 \\
    M\ped{3} & 7 & 7 & 6 \\
    M\ped{4} & 7 & 6 & 6 \\
    \hline
  \end{tabular}
  \caption{Tempo di lavoro delle macchine nella fabbrica 1}
\end{table}
\begin{table}[!htbp]
  \centering
  \begin{tabular}{ | c | c | c | c | }
    \hline
    \textbf{Macchina} & \textbf{8-16} & \textbf{16-24} & \textbf{24-8} \\
    \hline
    M\ped{1} & 7 & 6 & 7 \\
    M\ped{2} & 7 & 6 & 7 \\
    M\ped{3} & 7 & 6 & 7 \\
    \hline
  \end{tabular}
  \caption{Tempo di lavoro delle macchine nella fabbrica 2}
\end{table}

Per creare ciascun tipo di acciaio sono necessari tipi e quantità diverse tra loro di materiale grezzo, il cui costo totale in migliaia di euro è riportato per tonnellata nella tabella seguente:

\begin{table}[!htbp]
  \centering
  \begin{tabular}{ | c | c | c | c | }
    \hline
    & \textbf{Normale} & \textbf{Inox} & \textbf{Temprato} \\
    \hline
    Costo & 5.5 & 24 & 19 \\
    \hline
  \end{tabular}
  \caption{Costo totale in migliaia di euro per tonnellata di acciaio}
\end{table}


L'azienda a fine giornata trasporta tutto l'acciaio lavorato dalle due fabbriche verso i tre depositi di stoccaggio, utilizzando dei camion, la cui somma dei costi di benzina, usura e tempo sono riportati nelle seguenti tabelle:

\begin{table}[!htbp]
  \centering
  \begin{tabular}{ | c | c | c | c |}
    \hline
    \textbf{Fabbrica} & \textbf{D\ped{1}} & \textbf{D\ped{2}} & \textbf{D\ped{3}}\\
    \hline
    F\ped{1} & 1000 & 1340 & 2500 \\
    F\ped{2} & 1000 & 1900 & 1200 \\
    \hline
  \end{tabular}
  \caption{Costi in euro per trasportare un carico di 5 barili}
\end{table}

\begin{table}[!htbp]
  \centering
  \begin{tabular}{ | c | c | c | c |}
    \hline
    \textbf{Fabbrica} & \textbf{D\ped{1}} & \textbf{D\ped{2}} & \textbf{D\ped{3}}\\
    \hline
    F\ped{1} & 1 & 2 & 2.5 \\
    F\ped{2} & 1.8 & 1.9 & 2.1 \\
    \hline
  \end{tabular}
  \caption{Tempo necessario per trasportare un carico di 5 barili}
\end{table}

La richiesta di acciaio di ciascun tipo per ogni deposito è riassunta nella seguente tabella:

\begin{table}[!htbp]
  \centering
  \begin{tabular}{ | c | c | c | c | }
    \hline
    \textbf{Acciaio} & \textbf{D\ped{1}} & \textbf{D\ped{2}} & \textbf{D\ped{3}}  \\
    \hline
    Normale & 15 & 15 & 15 \\
    Inox & 10 & 8 & 9 \\
    Temprato & 10 & 10 & 12 \\
    \hline
  \end{tabular}
  \caption{Richiesta di acciaio per ciascun deposito}
\end{table}

Infine si assuma che tutto l'acciaio prodotto venga inviato ai depositi in quanto nelle fabbriche non c'è la possibilità di immagazzinare prodotto e che tutto l'acciaio lavorato venga assorbito dal mercato fruttando all'azienda i seguenti profitti in migliaia di euro per tonnellata:

\begin{table}[!htbp]
  \centering
  \begin{tabular}{ | c | c | c | c | }
    \hline
    & \textbf{Normale} & \textbf{Inox} & \textbf{Temprato}  \\
    \hline
    Profitto & 6 & 25 & 20 \\
    \hline
  \end{tabular}
  \caption{Profitti in migliaia di euro per tonnellata di acciaio}
\end{table}

Si formuli un modello di Programmazione Lineare che verifichi le specifiche del problema descritto, al fine massimizzare il profitto della fabbrica sapendo:

\begin{itemize}
\item Per cercare di diminuire i costi, le macchine possono produrre solo multipli di tonnellate di acciaio lavorato.
\item Per far funzionare una macchina servono 5 operai la cui retribuzione è 10 euro l'ora a ciascuno;
\item Gli operai per motivi di sicurezza non possono cambiare macchina durante il loro turno di lavoro e anche se la macchina non produce tutto il tempo vengono comunque pagati.
\item Un camion è guidato da un solo operaio la cui retribuzione oraria ammonta a 12 euro;
\item Un camion può trasportare al massimo 5 barili da due tonnellate l'uno di acciaio lavorato;
\item Se una macchina lavora più tipi di acciaio diversi tra loro, prima del cambio necessita di svolgere un ciclo di pulizia che costa 150 euro e impiega 5 operai per 1 ora;
\item Alla fine di ogni turno di lavoro occorre effettuare una pulizia della macchina se è stata adoperata;
\item Poiché è possibile impostare una macchina per produrre un solo tipo di prodotto per tutto il giorno, permettendo di non eseguire la pulitura della macchina alla fine di ogni turno, ma solo a fine giornata, valutare per entrambe le fabbriche se questa opzione permette un profitto maggiore e in caso indicare il prodotto e la macchina.
\item Nella prima fabbrica, la macchina M\ped{1} può lavorare solo acciaio Inox.
\item Nella seconda fabbrica, la macchina M\ped{2} può lavorare solo acciaio Temprato.
\item I depositi sono completamente vuoti e possono immagazzinare al massimo 40 tonnellate ciascuno di acciaio.
\end{itemize}
\newpage



\section{Soluzione}
Di seguito sarà data una spiegazione della soluzione

\subsection{Dati del modello}
	Il file .dat contiene una rappresentazione di tutti i dati e vincoli imposti dal problema.

	\subsubsection{Dichiarazione degli insiemi}
		In tale file vengono immediatamente definiti gli insiemi e si fa notare che quelli dei prodotti e delle macchine sono uguali per entrambe le fabbriche, anche se nella seconda non esiste la macchina M\ped{4} e non può produrre acciaio Inox.
		\begin{lstlisting}
		set TURNI := "8-16","16-24","24-8"; #Turni lavorativi imposti dal sindacato
		set PRODOTTI := "Normale","Inox","Temprato"; #Tipologia di acciaio
		set MACCHINE := "M1","M2","M3","M4"; #Macchine di lavoro
		set DEPOSITI := "D1","D2","D3"; #Depositi dell'azienda
		set FABBRICHE := "F1","F2"; #Fabbriche dell'azienda
		\end{lstlisting}

	\subsubsection{Definizione dei parametri}
		Successivamente vengono assegnati i valori ai parametri che verranno poi usati nel file .mod.
		Il parametro \textit{bigM} rappresenta un valore grande non raggiungibile dal problema, e viene usata per legare le variabili logiche alla quantità di prodotto creato. In questo caso il suo valore è 100 e cio\'e supera le tonnellate massime di acciaio lavorato che si pu\'o produrre in qualsiasi turno.
		\begin{lstlisting}
		param bigM := 100;
		\end{lstlisting}

		Il parametro \textit{guadagniAFinito} rappresenta il guadagno derivante dalla vendita di una tonnellata di acciaio lavorato sul mercato, mentre il parametro \textit{costiAGrezzo} rappresenta il costo dell'acciaio grezzo anch'esso per tonnellata.
		\begin{lstlisting}
		param guadagniAFinito :=
		      "Normale" 6000
		      "Inox" 25000
		      "Temprato" 20000 ;

		param costiAGrezzo := 
		      "Normale" 5500
		      "Inox" 24000
		      "Temprato" 19000 ;
		\end{lstlisting}

		Il parametro \textit{tempiMaxMacchine} rappresenta il tempo massimo di lavoro per ogni macchina in ogni turno e fabbrica. 
		Poich\'e nella fabbrica F\ped{2} non esiste la macchina M\ped{4} allora il tempo di produzione è impostato a 0.
		\begin{lstlisting}
		param tempiMaxMacchine:=	
		      ["F1",*,*]: "8-16" "16-24" "24-8" :=
		              "M1" 7 6 6
		      	      "M2" 7 7 6
		      	      "M3" 7 7 6
		      	      "M4" 7 6 6 

		      ["F2",*,*]: "8-16" "16-24" "24-8" :=
		      	      "M1" 7 6 7
		      	      "M2" 7 6 7
		      	      "M3" 7 6 7
			          "M4" 0 0 0;
		\end{lstlisting}

		Il parametro \textit{tempiProd} rappresenta il tempo di produzione di una tonnellata di acciaio, data una macchina, un turno e una fabbrica.
		Poich\'e nella fabbrica F\ped{2} non è possibile lavorare l'acciaio Inox nemmeno nelle macchine M\ped{1}, M\ped{2}, M\ped{3}, il valore corrispondente è molto alto e in questo caso corrisponde al valore del parametro \textit{bigM}.
		Nella fabbrica F\ped{2} i valori relativi alla macchina M\ped{4} potrebbero assumere qualsiasi altro valore, in quanto il parametro \textit{tempiMaxMacchine} farà si che non venga creato nessun prodotto da tale macchina.
		\begin{lstlisting}
		param tempiProd:=
		["F1",*,"8-16",*]: "Normale" "Inox" "Temprato" :=
		      "M1" 0.9 1.5 1.1
		      "M2" 1 1.6 1
		      "M3" 0.9 1.4 1.1
		      "M4" 0.8 1.4 1.1
		      
		["F1",*,"16-24",*]: "Normale" "Inox" "Temprato" :=
		      "M1" 1 1.5 1
		      "M2" 1 1.5 1
		      "M3" 0.9 1.5 1
		      "M4" 0.9 1.3 1

		["F1",*,"24-8",*]: "Normale" "Inox" "Temprato" :=
		      "M1" 1.3 1.7 1.4
		      "M2" 1.2 1.7 1.3
		      "M3" 1 1.3 1.2
		      "M4" 1 1.3 1.1

		["F2",*,"8-16",*]: "Normale" "Inox" "Temprato" :=
		      "M1" 1 100 1.2
		      "M2" 1 100 1.3
		      "M3" 1 100 1.1 
		      "M4" 100 100 100
		      
		["F2",*,"16-24",*]: "Normale" "Inox" "Temprato" :=
		      "M1" 1.1 100 1
		      "M2" 0.8 100 1.3
		      "M3" 0.9 100 1.3 
		      "M4" 100 100 100

		["F2",*,"24-8",*]: "Normale" "Inox" "Temprato" :=
		      "M1" 1 100 1
		      "M2" 1.2 100 1.1
		      "M3" 1 100 1 
		      "M4" 100 100 100;
		\end{lstlisting}

		Questi parametri rappresentano rispettivamente il numero di operai necessari ad ogni macchina per funzionare correttamente, il costo e il tempo necessario per pulirla, e  il salario orario di ogni dipendente che segue la produzione.
		\begin{lstlisting}
		param numOperaiMacchine := 5;
		param costoPuliziaM := 150;
		param tempoPuliziaM := 1;
		param costoOperaioProduzione := 10;
		\end{lstlisting}

		Il parametro \textit{mImpostateObbligatorie} permette di indicare per ogni fabbrica, macchina e tipo di produzione, se tale macchina deve essere impostata alla produzione del solo prodotto indicato.
		Questo parametro ha solamente valori binari e dunque per indicare che una macchina è impostata occorre usare il valore 1, altrimenti 0.
		In questo caso la macchina M\ped{1} produce solo acciaio Normale nella fabbrica F\ped{1}, mentre la macchina M\ped{2} produce solo acciaio Temprato nella fabbrica F\ped{2}
		Le fabbriche rimanenti possono produrre più tipi di acciaio o anche nessuno.
		\begin{lstlisting}
		param mImpostateObbligatorie :=
		      ["F1",*,*]: "Normale" "Inox" "Temprato" :=
		      	"M1" 1 0 0
			      "M2" 0 0 0 
			      "M3" 0 0 0 
			      "M4" 0 0 0 

		      ["F2",*,*]: "Normale" "Inox" "Temprato" :=
		            "M1" 0 0 0
			      "M2" 0 0 1 
			      "M3" 0 0 0 
			      "M4" 0 0 0; 
		\end{lstlisting}

		I parametri seguenti indicano rispettivamente i costi e i tempi di trasporto di un camion verso i depositi, il costo orario di un dipendente camionista e le tonnellate massime di prodotto che si possono trasportare.
		\begin{lstlisting}
		param costiTrasporto: "D1" "D2" "D3" := 
		      "F1" 1000 1340 2500
		      "F2" 1000 1900 1200;

		param tempiTrasporto: "D1" "D2" "D3" := 
		      "F1" 1 2.0 2.5
		      "F2" 1.8 1.9 2.1;

		param costoOperaioTrasporto := 12;
		param tonnellatePerCamion := 10; #5 barili da due tonnellate ciascuno
		\end{lstlisting}

		In fine i parametri \textit{richiesteMinDepositi} e \textit{richiesteMaxDepositi} rappresentano rispettivamente la richiesta di ogni tipo di acciaio da parte dei depositi e la loro capienza massima misurata in tonnellate di acciaio.
		\begin{lstlisting}
		param richiesteMinDepositi: "D1" "D2" "D3" :=
		      "Normale" 15 15 15
		      "Inox" 10 8 9
		      "Temprato" 10 10 12;

		param richiesteMaxDepositi:=
			"D1" 40
			"D2" 40
			"D3" 40;
		\end{lstlisting}



\newpage
	\subsection{Modello del problema}
	Il file .mod rappresenta il modello del problema.

	\subsubsection{Dichiarazione degli insiemi}
		Inizia dichiarando la presenza di insiemi che corrispondono a quelli presenti nel file .dat.
		\begin{lstlisting}
		set TURNI; 
		set PRODOTTI; 
		set MACCHINE;
		set DEPOSITI; 
		set FABBRICHE; 
		\end{lstlisting}

	\subsubsection{Dichiarazione dei parametri}
		Successivamente vengono definiti i parametri anch'essi presenti nel file .dat specificando, se necessario, gli insiemi da utilizzare per scorrere i valori.
		\begin{lstlisting}
		param bigM;
		param guadagniAFinito{PRODOTTI}; 
		param costiAGrezzo{PRODOTTI};
		param tempiMaxMacchine{FABBRICHE,MACCHINE,TURNI};
		param tempiProd{FABBRICHE,MACCHINE,TURNI,PRODOTTI}; 
		param numOperaiMacchine; 
		param costoPuliziaM; 
		param tempoPuliziaM;
		param costoOperaioProduzione;
		param mImpostateObbligatorie{FABBRICHE,MACCHINE,PRODOTTI};
		param costiTrasporto{FABBRICHE,DEPOSITI};
		param tempiTrasporto{FABBRICHE,DEPOSITI}; 
		param costoOperaioTrasporto;
		param tonnellatePerCamion;
		param richiesteMinDepositi{PRODOTTI,DEPOSITI}; 
		param richiesteMaxDepositi{DEPOSITI}; 
		\end{lstlisting}

	\subsubsection{Dichiarazione delle variabili}
		Successivamente vengono dichiarate le variabili del problema.

		La variabile \textit{al}(Acciaio Lavorato) rappresenta per ogni fabbrica, macchina e turni, la quantità di un preciso tipo di acciaio che è stata lavorata. I valori di questa variabile sono interi positivi o nulli.
		\begin{lstlisting}
		var al{FABBRICHE,MACCHINE,TURNI,PRODOTTI} integer >= 0;
		\end{lstlisting}

		La variabile \textit{at}(Acciaio Trasportato) rappresenta per ogni fabbrica la quantità di un preciso acciaio lavorato che deve essere trasportata verso gli altri depositi.
		Si noti che essendo la richiesta dei depositi e la produzione di ogni macchina in tonnellate, cio\'e un numero intero, allora anche questa variabile avr\'a valori interi positivi o nulli anche se non esplicitamente dichiarata come tale.
		\begin{lstlisting}
		var at{FABBRICHE,PRODOTTI,DEPOSITI} >= 0;
		\end{lstlisting}

		La variabile \textit{nc}(Numero Camion) rappresenta per ogni fabbrica e deposito, il numero di camion necessari per trasportare tutto il prodotto lavorato. I valori di questa variabile sono definiti interi positivi o nulli.
		\begin{lstlisting}
		var nc{FABBRICHE,DEPOSITI} integer >= 0;
		\end{lstlisting}

		La variabile logica \textit{mAcciaio} per ogni fabbrica, macchina,turno di lavoro e prodotto, indica se \'e stata eseguita una lavorazione del prodotto indicato. I valori di questa variabile sono dichiarati binari e dunque possono essere 1, se è stata eseguita la lavorazione del prodotto o 0 nel caso in cui la macchina non abbia eseguito la lavorazione.
		\begin{lstlisting}
		var mAcciaio{FABBRICHE,MACCHINE,TURNI,PRODOTTI} binary;
		\end{lstlisting}

		La variabile logica \textit{mAttive}(Macchine Attive) per ogni fabbrica, macchina e turno di lavoro, indica se è stata eseguita una produzione di acciaio. I valori di questa variabile sono dichiarati binari e dunque possono essere 1, se la macchina ha eseguito almeno una lavorazione, altrimenti 0.
		\begin{lstlisting}
		var mAttive{FABBRICHE,MACCHINE,TURNI} binary;
		\end{lstlisting}

		La variabile logica \textit{mImpostate}(Macchine Impostate) indica se una precisa macchina è stata impostata per la lavorazione di un preciso prodotto, data una fabbrica, una macchina ed un tipo di acciaio.I valori di questa variabile sono dichiarati binari e dunque possono essere 1, se la macchina è impostata su un preciso prodotto, altrimenti 0.
		\begin{lstlisting}
		var mImpostate{FABBRICHE,MACCHINE,PRODOTTI} binary;
		\end{lstlisting}

	\subsubsection{Funzione obbiettivo}
		La funzione obbiettivo vuole massimizzare il profitto e dunque calcola il guadagno netto proveniente dalla vendita di tutto l'acciaio prodotto meno la spesa di quello grezzo, utilizzato durante la lavorazione.
		\begin{lstlisting}
		sum{p in PRODOTTI}((guadagniAFinito[p]-costiAGrezzo[p])*sum{f in FABBRICHE,m in MACCHINE,t in TURNI}al[f,m,t,p])-
		\end{lstlisting}

		Successivamente a questo profitto viene sottratto il numero di dipendenti necessari per far funzionare le macchine moltiplicato per il salario orario e il numero di ore lavorative.
		\begin{lstlisting}
		sum{f in FABBRICHE, m in MACCHINE,t in TURNI}(mAttive[f,m,t]*numOperaiMacchine*tempiMaxMacchine[f,m,t]*costoOperaioProduzione)-
		\end{lstlisting}

		Inoltre viene sottratto il costo derivante dalle pulizie delle macchine non impostate.
		Si noti che una macchina che produce acciaio, ma non è stata impostata, occorre pulirla ad ogni lavorazione di acciaio differente e ad ogni fine turno, mentre una macchina se impostata allora produrr\'a per tutti i turni un solo tipo di acciaio.
		\begin{lstlisting}
		sum{f in FABBRICHE, m in MACCHINE,t in TURNI}sum{p in PRODOTTI}(mAcciaio[f,m,t,p]- mImpostate[f,m,p])*costoPuliziaM-
		\end{lstlisting}

		Successivamente viene sottratto anche il costo derivante dalle pulizie a fine giornata di ogni macchina impostata.
		\begin{lstlisting}
		sum{f in FABBRICHE, m in MACCHINE,p in PRODOTTI}mImpostate[f,m,p]*costoPuliziaM -
		\end{lstlisting}

		In fine viene sottratto anche il costo sia del camion che del personale che deriva dal trasporto dell'acciaio dalle fabbriche ai depositi .
		\begin{lstlisting}
		sum{f in FABBRICHE,d in DEPOSITI}(nc[f,d]*costiTrasporto[f,d])-
		sum{f in FABBRICHE,d in DEPOSITI}(nc[f,d]*tempiTrasporto[f,d])*costoOperaioTrasporto;
		\end{lstlisting}


	\subsubsection{Vincoli}
		In fine si trovano i vincoli che impongono limiti o precisi valori alle variabili.

		Il vincolo \textit{VtempoMNImpostate}(Vincolo di tempo macchine non impostate) fa si che per ogni fabbrica, macchina e turno, il tempo effettivo di lavorazione dell'acciaio che viene prodotto, sia sempre inferiore o uguale al tempo di lavoro dei dipendenti imposto dai sindacati.
		In particolar modo tiene conto anche del tempo di pulizia della macchina solo se essa non è impostata.
		\textit{(mAcciaio[f,m,t,p]- mImpostate[f,m,p])} risulta 1 solo quando la macchina non è impostata, altrimenti 0.
		\begin{lstlisting}
		s.t. VtempoMNImpostate{f in FABBRICHE, m in MACCHINE,t in TURNI}: sum{p in PRODOTTI}(tempiProd[f,m,t,p]*al[f,m,t,p] + (mAcciaio[f,m,t,p]- mImpostate[f,m,p])*tempoPuliziaM) <= tempiMaxMacchine[f,m,t];
		\end{lstlisting}

		Questo vincolo fa si che per ogni fabbrica, macchina e prodotto, il tempo complessivo di lavorazione sia sempre inferiore o uguale al tempo di lavoro dei dipendenti imposto dai sindacati.
		In particolar modo tiene conto del tempo di pulizia della macchina nel caso essa sia impostata alla produzione di un solo tipo di acciaio.
		\begin{lstlisting}
		s.t. VtempoMImpostate{f in FABBRICHE, m in MACCHINE,p in PRODOTTI}: sum{t in TURNI}(tempiProd[f,m,t,p]*al[f,m,t,p])+mImpostate[f,m,p]*tempoPuliziaM <= sum{t in TURNI}tempiMaxMacchine[f,m,t];
		\end{lstlisting}

		Questo vincolo lega la variabile \textit{al} con la variabile \textit{at}, permettendo così di legare le richieste di acciaio dei depositi, con la loro produzione.
		\begin{lstlisting}
		s.t. VCat{f in FABBRICHE,p in PRODOTTI}: sum{m in MACCHINE,t in TURNI}al[f,m,t,p] = sum{d in DEPOSITI}at[f,p,d];
		\end{lstlisting}

		Questi due vincoli impongono alla variabile \textit{at} un limite inferiore dovuto alla richiesta dei depositi e un vincolo superiore dovuto alla loro massima capienza
		\begin{lstlisting}
		s.t. VRichiesteMinDepositi{p in PRODOTTI,d in DEPOSITI}: sum{f in FABBRICHE}at[f,p,d] >= richiesteMinDepositi[p,d];
		s.t. VRichiesteMaxDepositi{d in DEPOSITI}: sum{f in FABBRICHE,p in PRODOTTI}at[f,p,d] <= richiesteMaxDepositi[d];
		\end{lstlisting}

		Questo vincolo lega la variabile \textit{nc} a quella \textit{at}, permettendo così di definire per ogni fabbrica e deposito, il numero di camion necessari per trasportare tutto l'acciaio prodotto.
		In particolare la variabile \textit{nc} è un intero e poich\'e a parità di acciaio, l'aumentare del numero di camion diminuisce il profitto, in automatico il numero sarà l'intero superiore risultante dalla divisione della \textit{somma del prodotto diviso la capienza di un camion}
		\begin{lstlisting}
		s.t. VUCamion{f in FABBRICHE,d in DEPOSITI}: sum{p in PRODOTTI}at[f,p,d] <= nc[f,d]*tonnellatePerCamion;
		\end{lstlisting}

		Questo vincolo permette di legare la variabile logica \textit{mAcciaio} alla variabile al
		Poich\'e all'aumentare \textit{mAcciaio} provoca un costo maggiore, non si presenteranno valori spuri in quanto la variabile verr\'a posta in automatico a 0 quando possibile
		\begin{lstlisting}
		s.t. VUmAcciaio{f in FABBRICHE,m in MACCHINE,t in TURNI,p in PRODOTTI}: al[f,m,t,p] <= bigM*mAcciaio[f,m,t,p];
		\end{lstlisting}

		Questo vincolo permette di legare la variabile logica mAttive alla variabile \textit{mAcciaio}
		Poich\'e all'aumentare mAttive provoca un costo maggiore, non si presenteranno valori spuri in quanto la variabile verrà posta in automatico a zero quando possibile, altrimenti viene posta a 1.
		\begin{lstlisting}
		s.t. VUmAttive{f in FABBRICHE,m in MACCHINE,t in TURNI}: sum{p in PRODOTTI}mAcciaio[f,m,t,p] <= bigM*mAttive[f,m,t];
		\end{lstlisting}

		Questo vincolo permette di legare la variabile logica \textit{mImpostate} alla variabile logica \textit{mAcciaio}.
		Data una fabbrica, una macchina e un prodotto, la macchina risulta impostata su tale lavorazione se e solo se non ha lavorato altri prodotti in tutta la giornata.
		Poich\'e impostare una macchina induce ad eseguire meno cicli di pulizia, e dunque ad un minor costo, la funzione obbiettivo farà si che non si verifichino valori spuri, e imposterà ad 1 la variabile logica mImpostate quando possibile.
		In particolare \textit{mImpostate[f,m,p] + 1 - 2*mImpostate[f,m,p]} vale 0 se mImpostate[f,m,p] = 1 e 0 se mImpostate[f,m,p] = 0.
		\begin{lstlisting}
		s.t. VUmImpostate{f in FABBRICHE,m in MACCHINE,p in PRODOTTI}:  sum{t in TURNI,pp in PRODOTTI:pp <> p}mAcciaio[f,m,t,pp] <= bigM*(mImpostate[f,m,p] + 1 - 2*mImpostate[f,m,p]) ;
		\end{lstlisting}

		Questo vincolo fa si di limitare il problema delle macchine da impostare, imponendo dei vincoli esterni su esso tramite il parametro passato dal file .dat.
		\begin{lstlisting}
		s.t. VmImpostateObbligatorie{f in FABBRICHE,m in MACCHINE,p in PRODOTTI}:mImpostate[f,m,p] >= mImpostateObbligatorie[f,m,p];
		\end{lstlisting}

		Questo vincolo fa si che data una fabbrica e una macchina, la variabile logica \textit{mImpostate} possa risultare 1 solo su un unico prodotto.
		\begin{lstlisting}
		s.t. VCmImpostate{f in FABBRICHE,m in MACCHINE}: sum{p in PRODOTTI}mImpostate[f,m,p] <= 1;
		\end{lstlisting}



		\subsubsection{Soluzione del modello}
		Tramite il comando:
		\begin{lstlisting}
			reset; model steel.mod; data steel.dat;option solver cplexamp; solve;
		\end{lstlisting}
		Si ottiene la seguente soluzione per la funzione obbiettivo: \textbf{62062.4}\\
		E i seguenti valori per le variabili del problema primale:

		Di seguito la variabile \textit{al} mostra il numero di tonnellate che si devono produrre acciaio, per ogni fabbrica,  macchina,turno e prodotto. 
		\begin{lstlisting}
			al [F1,*,16-24,*]
			:  Inox Normale Temprato    :=
			M1   0      6       0
			M2   4      0       0
			M3   4      0       0
			M4   4      0       0

			 [F1,*,24-8,*]
			:  Inox Normale Temprato    :=
			M1   0      4       0
			M2   0      4       0
			M3   4      0       0
			M4   4      0       0

			 [F1,*,8-16,*]
			:  Inox Normale Temprato    :=
			M1   0      7       0
			M2   0      6       0
			M3   5      0       0
			M4   5      0       0

			 [F2,*,16-24,*]
			:  Inox Normale Temprato    :=
			M1   0      0       6
			M2   0      0       4
			M3   0      5       0
			M4   0      0       0

			 [F2,*,24-8,*]
			:  Inox Normale Temprato    :=
			M1   0      0       7
			M2   0      0       6
			M3   0      7       0
			M4   0      0       0

			 [F2,*,8-16,*]
			:  Inox Normale Temprato    :=
			M1   0      0       5
			M2   0      0       5
			M3   0      7       0
			M4   0      0       0;
		\end{lstlisting}

		Di seguito sono elencate tutte le tonnellate di ogni tipo di acciaio che sono state prodotte da una fabbrica e devono essere trasportate con dei camion verso i depositi.
		\begin{lstlisting}
			at :=
			F1 Inox     D1   10
			F1 Inox     D2   11
			F1 Inox     D3    9
			F1 Normale  D1   10
			F1 Normale  D2   16
			F1 Normale  D3    1
			F1 Temprato D1    0
			F1 Temprato D2    0
			F1 Temprato D3    0
			F2 Inox     D1    0
			F2 Inox     D2    0
			F2 Inox     D3    0
			F2 Normale  D1    5
			F2 Normale  D2    0
			F2 Normale  D3   14
			F2 Temprato D1   10
			F2 Temprato D2   10
			F2 Temprato D3   13;
		\end{lstlisting}

		Di seguito sono elencati i numeri di camion necessari per trasportare tutto il prodotto verso i depositi.
		\begin{lstlisting}
			nc :=
			F1 D1   2
			F1 D2   3
			F1 D3   1
			F2 D1   2
			F2 D2   1
			F2 D3   3;
		\end{lstlisting}

		Di seguito sono indicate per ogni fabbrica,turno e macchina tutte le produzioni che sono state eseguite.
		\begin{lstlisting}
			mAcciaio [F1,*,16-24,*]
			:  Inox Normale Temprato    :=
			M1   0      1       0
			M2   1      0       0
			M3   1      0       0
			M4   1      0       0

			 [F1,*,24-8,*]
			:  Inox Normale Temprato    :=
			M1   0      1       0
			M2   0      1       0
			M3   1      0       0
			M4   1      0       0

			 [F1,*,8-16,*]
			:  Inox Normale Temprato    :=
			M1   0      1       0
			M2   0      1       0
			M3   1      0       0
			M4   1      0       0

			 [F2,*,16-24,*]
			:  Inox Normale Temprato    :=
			M1   0      0       1
			M2   0      0       1
			M3   0      1       0
			M4   0      0       0

			 [F2,*,24-8,*]
			:  Inox Normale Temprato    :=
			M1   0      0       1
			M2   0      0       1
			M3   0      1       0
			M4   0      0       0

			 [F2,*,8-16,*]
			:  Inox Normale Temprato    :=
			M1   0      0       1
			M2   0      0       1
			M3   0      1       0
			M4   0      0       0;
		\end{lstlisting}

		Come si pu\'o notare tutte le macchine delle due fabbriche, in tutti i turni producono qualche tipo di acciaio.
		\begin{lstlisting}
			mAttive [F1,*,*]
			:  16-24 24-8 8-16    :=
			M1    1    1    1
			M2    1    1    1
			M3    1    1    1
			M4    1    1    1

			 [F2,*,*]
			:  16-24 24-8 8-16    :=
			M1    1    1    1
			M2    1    1    1
			M3    1    1    1
			M4    0    0    0;
		\end{lstlisting}

		In fine risulta essere conveniente impostare le macchine M\ped{3} e M\ped{4} della fabbrica F\ped{1} alla produzione di acciaio Inox e la macchina M\ped{1} e M\ped{3} della fabbrica F\ped{2} alla produzione rispettivamente di acciaio Temprato e Normale.
		\begin{lstlisting}
			mImpostate [F1,*,*]
			:  Inox Normale Temprato    :=
			M1   0      1       0
			M2   0      0       0
			M3   1      0       0
			M4   1      0       0

			 [F2,*,*]
			:  Inox Normale Temprato    :=
			M1   0      0       1
			M2   0      0       1
			M3   0      1       0
			M4   0      0       0;
		\end{lstlisting}


		Inoltre si ottiene i seguenti valori ottimali per le variabili del problema duale: 
		\begin{lstlisting}
			VtempoMNImpostate [F1,*,*]
			:  16-24 24-8 8-16    :=
			M1    0    0    0
			M2    0    0    0
			M3    0    0    0
			M4    0    0    0

			 [F2,*,*]
			:  16-24 24-8 8-16    :=
			M1    0    0    0
			M2    0    0    0
			M3    0    0    0
			M4    0    0    0;

			VtempoMImpostate [F1,*,*]
			:  Inox Normale Temprato    :=
			M1   0      0       0
			M2   0      0       0
			M3   0      0       0
			M4   0      0       0

			 [F2,*,*]
			:   Inox Normale Temprato    :=
			M1     0      0       0
			M2     0      0       0
			M3     0      0       0
			M4   300    300     300;

			VCat :=
			F1 Inox       0
			F1 Normale    0
			F1 Temprato   0
			F2 Inox       0
			F2 Normale    0
			F2 Temprato   0;

			VRichiesteMinDepositi :=
			Inox     D1   0
			Inox     D2   0
			Inox     D3   0
			Normale  D1   0
			Normale  D2   0
			Normale  D3   0
			Temprato D1   0
			Temprato D2   0
			Temprato D3   0;

			VRichiesteMaxDepositi [*] :=
			D1  0
			D2  0
			D3  0;

			VUCamion :=
			F1 D1   0
			F1 D2   0
			F1 D3   0
			F2 D1   0
			F2 D2   0
			F2 D3   0;

			VUmAcciaio [F1,*,16-24,*]
			:    Inox Normale Temprato    :=
			M1   1000     0      1000
			M2      0     0         0
			M3      0     0         0
			M4      0     0         0

			 [F1,*,24-8,*]
			:    Inox Normale Temprato    :=
			M1   1000     0      1000
			M2      0     0         0
			M3      0     0         0
			M4      0     0         0

			 [F1,*,8-16,*]
			:    Inox Normale Temprato    :=
			M1   1000     0      1000
			M2      0     0         0
			M3      0     0         0
			M4      0     0         0

			 [F2,*,16-24,*]
			:    Inox Normale Temprato    :=
			M1      0      0      0
			M2   1000    500      0
			M3      0      0      0
			M4      0      0      0

			 [F2,*,24-8,*]
			:    Inox Normale Temprato    :=
			M1      0      0      0
			M2   1000    500      0
			M3      0      0      0
			M4      0      0      0

			 [F2,*,8-16,*]
			:    Inox Normale Temprato    :=
			M1      0      0      0
			M2   1000    500      0
			M3      0      0      0
			M4      0      0      0;

			VUmAttive [F1,*,*]
			:  16-24 24-8 8-16    :=
			M1    0    0    0
			M2    0    0    0
			M3    0    0    0
			M4    0    0    0

			 [F2,*,*]
			:  16-24 24-8 8-16    :=
			M1    0    0    0
			M2    0    0    0
			M3    0    0    0
			M4    0    0    0;

			VUmImpostate [F1,*,*]
			:  Inox Normale Temprato    :=
			M1   0    99850     0
			M2   0        0     0
			M3   0        0     0
			M4   0        0     0

			 [F2,*,*]
			:  Inox Normale Temprato    :=
			M1   0      0         0
			M2   0      0     99850
			M3   0      0         0
			M4   0      0         0;

			VmImpostateObbligatorie [F1,*,*]
			:  Inox   Normale  Temprato    :=
			M1   0    -9985000     0
			M2   0           0     0
			M3   0           0     0
			M4   0           0     0

			 [F2,*,*]
			:  Inox Normale   Temprato    :=
			M1   0      0            0
			M2   0      0     -9985000
			M3   0      0            0
			M4   0      0            0;

			VCmImpostate :=
			F1 M1   300
			F1 M2     0
			F1 M3     0
			F1 M4     0
			F2 M1     0
			F2 M2   300
			F2 M3     0
			F2 M4     0;

			VtempoMNImpostate [F1,*,*]
			:  16-24 24-8 8-16    :=
			M1    0    0    0
			M2    0    0    0
			M3    0    0    0
			M4    0    0    0

			 [F2,*,*]
			:  16-24 24-8 8-16    :=
			M1    0    0    0
			M2    0    0    0
			M3    0    0    0
			M4    0    0    0;
		\end{lstlisting}



	\newpage

%----------------------------------------------------------------------------------------
% SOLUZIONE PROBLEMA
%----------------------------------------------------------------------------------------
\section{File}
Di seguito sono riportati i file .dat e .mod che rappresentano un modello della soluzione del problema

\subsection{File .dat}
\lstinputlisting{../amplFiles/steel.dat}

\newpage

\subsection{File .mod}
\lstinputlisting{../amplFiles/steel.mod}

\newpage

\subsection{Soluzione}
\lstinputlisting{../amplFiles/solution.txt}






\end{document}
